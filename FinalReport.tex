% Options for packages loaded elsewhere
\PassOptionsToPackage{unicode}{hyperref}
\PassOptionsToPackage{hyphens}{url}
\documentclass[
  11pt,
]{article}
\usepackage{xcolor}
\usepackage[margin=1in]{geometry}
\usepackage{amsmath,amssymb}
\setcounter{secnumdepth}{5}
\usepackage{iftex}
\ifPDFTeX
  \usepackage[T1]{fontenc}
  \usepackage[utf8]{inputenc}
  \usepackage{textcomp} % provide euro and other symbols
\else % if luatex or xetex
  \usepackage{unicode-math} % this also loads fontspec
  \defaultfontfeatures{Scale=MatchLowercase}
  \defaultfontfeatures[\rmfamily]{Ligatures=TeX,Scale=1}
\fi
\usepackage{lmodern}
\ifPDFTeX\else
  % xetex/luatex font selection
\fi
% Use upquote if available, for straight quotes in verbatim environments
\IfFileExists{upquote.sty}{\usepackage{upquote}}{}
\IfFileExists{microtype.sty}{% use microtype if available
  \usepackage[]{microtype}
  \UseMicrotypeSet[protrusion]{basicmath} % disable protrusion for tt fonts
}{}
\makeatletter
\@ifundefined{KOMAClassName}{% if non-KOMA class
  \IfFileExists{parskip.sty}{%
    \usepackage{parskip}
  }{% else
    \setlength{\parindent}{0pt}
    \setlength{\parskip}{6pt plus 2pt minus 1pt}}
}{% if KOMA class
  \KOMAoptions{parskip=half}}
\makeatother
\usepackage{graphicx}
\makeatletter
\newsavebox\pandoc@box
\newcommand*\pandocbounded[1]{% scales image to fit in text height/width
  \sbox\pandoc@box{#1}%
  \Gscale@div\@tempa{\textheight}{\dimexpr\ht\pandoc@box+\dp\pandoc@box\relax}%
  \Gscale@div\@tempb{\linewidth}{\wd\pandoc@box}%
  \ifdim\@tempb\p@<\@tempa\p@\let\@tempa\@tempb\fi% select the smaller of both
  \ifdim\@tempa\p@<\p@\scalebox{\@tempa}{\usebox\pandoc@box}%
  \else\usebox{\pandoc@box}%
  \fi%
}
% Set default figure placement to htbp
\def\fps@figure{htbp}
\makeatother
\setlength{\emergencystretch}{3em} % prevent overfull lines
\providecommand{\tightlist}{%
  \setlength{\itemsep}{0pt}\setlength{\parskip}{0pt}}
\usepackage{bookmark}
\IfFileExists{xurl.sty}{\usepackage{xurl}}{} % add URL line breaks if available
\urlstyle{same}
\hypersetup{
  pdftitle={Correlation between Inflation and Unemployment},
  pdfauthor={Edison Jiang, Vincent Chau, Yamato Lchii},
  hidelinks,
  pdfcreator={LaTeX via pandoc}}

\title{Correlation between Inflation and Unemployment}
\author{Edison Jiang, Vincent Chau, Yamato Lchii}
\date{2025/11/07}

\begin{document}
\maketitle

\section{Abstract}\label{abstract}

In this project, we examine the empirical relationship between the U.S.
inflation rate and the unemployment rate to assess the extent to which
contemporary data align with the classical Phillips curve. We use annual
macroeconomic data series from the Federal Reserve Economic Data (FRED)
and the World Bank, including U.S. CPI inflation, 10-year breakeven
inflation, and the unemployment rate. We first construct time series and
scatter plots for exploratory visualization of the data, followed by
calculating correlation coefficients and fitting regression models. In a
small sample of recent years, U.S. 10-year breakeven inflation and the
unemployment rate exhibit a strong and significant negative correlation
(r ≈ −0.87, p ≈ 0.02), consistent with the ``inflation--unemployment
trade-off'' described by the short-term Phillips curve. In contrast,
using the longer-term World Bank series of U.S. CPI inflation and
unemployment, the correlation is close to zero and statistically
insignificant (r ≈ 0.08, p ≈ 0.51), indicating that this relationship is
not stable over time. Log--log regressions of unemployment on inflation
show statistically significant elasticity, but the economic significance
is small, and the model's explanatory power is low (R² ≈ 0.06). Overall,
our results suggest that the negative link between inflation and
unemployment does appear in some subsamples but is not a robust or
permanent feature of U.S. data, highlighting the importance of sample
selection and model specification when using the Phillips curve for
policy analysis.

\subsection{Intro}\label{intro}

The relationship between inflation and unemployment is a central topic
in modern macroeconomic policy discussions. Since A. W. Phillips first
recorded a negative correlation between wage inflation and unemployment
in the UK, economists and policymakers have often used the so-called
`Phillips curve' to consider the short-term trade-offs faced by central
banks. Its most basic form suggests that periods of low unemployment are
often accompanied by higher inflation rates, whereas high unemployment
periods are associated with lower inflation rates. This view has guided
the formulation of stabilization policies for decades and continues to
influence discussions on interest rate setting and fiscal interventions.

However, historical events and recent experience have raised questions
about the stability and reliability of this relationship. The
stagflation of the 1970s, the low-inflation environment during the Great
Moderation, the Great Recession of 2008-2009, and the COVID-19 pandemic
have all shown that the combination of inflation and unemployment does
not always fit neatly along a downward-sloping curve. In particular,
before 2020, U.S. inflation remained low despite relatively low
unemployment, while after the pandemic, inflation surged significantly,
whereas the recovery of the unemployment rate was relatively slow. These
developments have led some observers to argue that the Phillips curve
has `flattened' or even `disappeared,' while others believe that it
remains valid when appropriate time horizons and modeling choices are
applied.

In this project, we use publicly available macroeconomic data to explore
whether the Phillips curve relationship is reflected in recent U.S.
economic conditions and how it compares to longer historical patterns.
Our primary focus is the United States, combining data from the Federal
Reserve Economic Data (FRED) database and the World Bank. FRED data
includes expected inflation rates, 10-year breakeven inflation rates,
and the non-employment rate. The World Bank provides longer annual
series for U.S. consumer price index (CPI) inflation and unemployment
rates. By studying both a smaller recent sample and a longer historical
sample, we can assess whether the sign, strength, and statistical
significance of the relationship between inflation and unemployment
differ depending on the period and inflation measure used.
Methodologically, we adopt a combination of graphical and econometric
tools.

First, we construct time series graphs and scatter plots to visually
illustrate the co-movement between inflation and unemployment. Then, we
calculate correlation coefficients and estimate simple regression
models, including a log-log regression that allows us to interpret the
slope as an elasticity. By comparing results across different datasets
and models, we can evaluate the extent to which U.S. data support a
Phillips curve-style trade-off and the degree of stability of this
relationship. The remainder of the report is organized as follows. The
next section provides a more detailed description of our data sources
and cleaning procedures. Then, we present our visualization results and
empirical analysis, and finally, we discuss our key findings and their
implications for interpreting the Phillips curve in contemporary policy
debates.

\subsection{Data}\label{data}

Our empirical analysis is based on publicly available macroeconomic data
from two primary sources: the Federal Reserve Economic Data (FRED)
database and the World Bank's World Development Indicators. Both sources
provide annual data series on U.S. inflation and unemployment. Using
multiple datasets for the same concepts helps us verify the consistency
of data across different sources and test the sensitivity of our results
to different measures of inflation and sample periods.

From the FRED database, we use the 10-year breakeven inflation rate and
the nonfarm unemployment rate. The 10-year breakeven inflation rate is
calculated as the yield difference between nominal U.S. Treasury bonds
and Treasury Inflation-Protected Securities (TIPS) and is commonly used
as a measure of the expected inflation rate over the next decade. The
nonfarm unemployment rate measures the percentage of the labor force
that is unemployed and actively seeking work. Since the original FRED
data are provided at a monthly frequency, we convert them to annual data
by taking simple yearly averages. Our sample includes all years for
which both series are available, but in recent periods, there are
relatively few observations per year.

From the World Bank's World Development Indicators, we obtain U.S.
annual Consumer Price Index (CPI) inflation rates and total unemployment
rates. CPI inflation represents the percentage change in the consumer
price index for all goods and services, while the unemployment rate
measures the proportion of the labor force that is without work but able
and actively looking for a job. These data series cover several decades,
providing a longer historical record than breakeven inflation data. We
focus only on the years for which both CPI inflation and unemployment
rates are available, resulting in a balanced annual dataset with several
dozen observations.

Before conducting the analysis, we perform a series of basic cleaning
and calibration steps. For each pair of variables, we merge the series
by year and remove observations with missing values for inflation or
unemployment. For FRED data, this leaves only a small number of recent
annual observations, which limits the precision and generalizability of
our estimates. For the longer World Bank sample, we also apply log
transformations to inflation and unemployment rates to estimate a
log-log regression model, allowing us to interpret slope coefficients as
elasticities. Overall, our cleaning process generates two primary
datasets: a shorter FRED sample with expected inflation and a longer
World Bank sample with actual CPI inflation, which together form the
basis for the visualizations and econometric analyses in the following
chapters.

\subsection{Visualizations}\label{visualizations}

\begin{verbatim}
## 
##  Pearson's product-moment correlation
## 
## data:  fred_data$Inflation and fred_data$Unemployment
## t = -3.5715, df = 4, p-value = 0.02334
## alternative hypothesis: true correlation is not equal to 0
## 95 percent confidence interval:
##  -0.9859350 -0.2087911
## sample estimates:
##        cor 
## -0.8725086
\end{verbatim}

\pandocbounded{\includegraphics[keepaspectratio]{FinalReport_files/figure-latex/unnamed-chunk-1-1.pdf}}

\begin{verbatim}
## 
##  Pearson's product-moment correlation
## 
## data:  wb_data$Inflation and wb_data$Unemployment
## t = 0.66103, df = 63, p-value = 0.511
## alternative hypothesis: true correlation is not equal to 0
## 95 percent confidence interval:
##  -0.1642286  0.3204082
## sample estimates:
##        cor 
## 0.08299491
\end{verbatim}

\pandocbounded{\includegraphics[keepaspectratio]{FinalReport_files/figure-latex/unnamed-chunk-1-2.pdf}}
Figure 1 shows a scatter plot of 10-year break-even inflation versus
civilian unemployment using FRED data. Each point represents one year in
the sample. The scatter plot slopes sharply downward, indicating that
years with higher unemployment tend to be accompanied by lower
break-even inflation, while years with lower unemployment tend to be
accompanied by higher expected inflation. The fitted line is quite
steep, and there are no significant outliers affecting this pattern,
suggesting a strong negative correlation between expected inflation and
unemployment in this recent small sample. Visually, the plot aligns with
the trade-off of the short-run Phillips curve, where a tight labor
market tends to be accompanied by higher inflation expectations.

Figure 2 replicates the above analysis using longer time-series data
from the World Bank, covering US real CPI inflation and total
unemployment. Compared to Figure 1, the points in this scatter plot are
more dispersed, and the fitted line is almost horizontal with only a
slight upward slope. Years with high CPI inflation may correspond to
high or low unemployment, and there is no obvious downward sloping
pattern, indicating that there is no strong Phillips curve relationship
between the two. As can be seen from Figure 2, a strong and stable
bivariate association between CPI inflation and unemployment has not
been observed over the longer historical period covered by the World
Bank data.

\pandocbounded{\includegraphics[keepaspectratio]{FinalReport_files/figure-latex/unnamed-chunk-2-1.pdf}}
Figure 3 compares the annual US unemployment rate from the two data
sources used in this project. The red line represents the civilian
unemployment rate series provided by FRED, and the blue line represents
the total unemployment rate reported by the World Bank for the same
period. Throughout the sample period, the two lines almost overlap, with
both series peaking during major recessions such as the early 1980s, the
Great Recession, and the recession triggered by the COVID-19 pandemic,
while declining during economic expansions. The differences between the
two indicators are small, typically within a few percentage points. This
visualization serves as a consistency test: it shows that the
unemployment rate series from FRED and the World Bank provide very
similar information, therefore the differences in our subsequent results
across different datasets are primarily due to the choice of inflation
indicators and sample periods, rather than differences in the method of
measuring the unemployment rate.

\begin{verbatim}
## 
## Call:
## lm(formula = log_fred_infl ~ log_fred_unem, data = fred_data)
## 
## Residuals:
##      Min       1Q   Median       3Q      Max 
## -0.31683 -0.04314 -0.00040  0.03553  0.24367 
## 
## Coefficients:
##               Estimate Std. Error t value Pr(>|t|)    
## (Intercept)   0.725515   0.016627  43.636  < 2e-16 ***
## log_fred_unem 0.075250   0.009619   7.823 1.45e-14 ***
## ---
## Signif. codes:  0 '***' 0.001 '**' 0.01 '*' 0.05 '.' 0.1 ' ' 1
## 
## Residual standard error: 0.08405 on 898 degrees of freedom
## Multiple R-squared:  0.0638, Adjusted R-squared:  0.06276 
## F-statistic:  61.2 on 1 and 898 DF,  p-value: 1.447e-14
\end{verbatim}

\pandocbounded{\includegraphics[keepaspectratio]{FinalReport_files/figure-latex/unnamed-chunk-3-1.pdf}}
Figure 4 visualizes the relationship between inflation and unemployment
based on FRED in logarithmic form. The horizontal axis represents the
logarithm of the civilian unemployment rate, and the vertical axis
represents the logarithm of the break-even inflation rate, with each
point corresponding to one year. The blue line represents the
logarithmic form of the fitted linear relationship. The regression line
still slopes downward, indicating that even after logarithmic
transformation, a higher unemployment rate is still associated with a
lower expected inflation rate. However, the broad gray confidence band
reflects the limited number of observations and significant fitting
noise. In other words, observing the data in logarithmic form reinforces
the negative correlation observed in Figure 1, while also indicating
that the estimated elasticity is not accurate in this small sample, and
therefore, relationships of the Phillips curve type should be
interpreted with caution.

\section{Analysis}\label{analysis}

In this section we complement our visual evidence with more formal
statistical analysis. We first quantify the strength and significance of
the bivariate relationships between inflation and unemployment using
correlation tests, and then estimate simple linear regression models to
assess whether unemployment can help predict inflation. Throughout, we
rely on standard R tools such as cov.test() and lm() (along with their
summary output and diagnostic plots) to obtain correlation coefficients,
p-values, and measures of goodness of fit.

For the FRED data, we consider the pair consisting of the annual 10-year
breakeven inflation rate and the civilian unemployment rate. The sample
correlation between these two series is strongly negative, consistent
with the downward-sloping pattern in Figure 1. A cov.test() of this pair
rejects the null hypothesis of zero correlation at conventional
significance levels, indicating that the negative association is
unlikely to be due to random chance in this small sample. In other
words, years with higher unemployment are systematically associated with
lower breakeven inflation, and the magnitude of the correlation is large
in absolute value. By contrast, when we apply the same procedure to the
World Bank series for CPI inflation and unemployment, the estimated
correlation is very close to zero and statistically insignificant, in
line with the diffuse cloud of points in Figure 2. In this longer
historical sample, we cannot reject the hypothesis that inflation and
unemployment are uncorrelated in a simple bivariate sense. As a
consistency check, we also compute the correlation between the FRED and
World Bank unemployment rates shown in Figure 3 and find it to be very
close to one and highly significant, confirming that discrepancies
across datasets in our results are driven by inflation measures and
sample periods rather than by differences in how unemployment is
measured.

We next investigate whether unemployment can be used to predict
inflation within a linear regression framework. For the FRED data, we
estimate a simple model in which annual breakeven inflation is regressed
on the civilian unemployment rate. The estimated slope coefficient is
negative and statistically different from zero, implying that an
increase in the unemployment rate is associated with a reduction in
expected inflation. The corresponding R-squared is relatively high for
this small sample, reflecting the strong negative co-movement between
the two variables seen in the scatterplot. At the same time, because the
regression is based on only a handful of annual observations, the
standard errors are sizable and the fitted line should be interpreted as
describing one particular period rather than a stable ``law'' of motion.

For the longer World Bank sample, we run an analogous regression of CPI
inflation on the unemployment rate. Here the estimated slope is small in
magnitude and not statistically significant, and the R-squared is very
low, indicating that unemployment explains only a tiny fraction of the
variation in realized CPI inflation across years. To explore potential
non-linearities and to go beyond the purely linear specifications
emphasized in class, we also estimate a log--log model using the FRED
data, where we regress the log of breakeven inflation on the log of the
unemployment rate. The fitted line in Figure 4 still slopes downward,
and the estimated elasticity of inflation with respect to unemployment
is negative but modest in size, meaning that a given percentage increase
in unemployment is associated with a relatively small percentage
decrease in expected inflation. The confidence band around the
regression line is wide, again reflecting the limited amount of data and
suggesting that the elasticity is estimated imprecisely.

Finally, our results point to important time-dependence in the
inflation--unemployment relationship. In the short recent period
captured by the breakeven data, both the correlation tests and the
linear models yield a clear negative association that resembles a
traditional short-run Phillips curve. In the longer World Bank sample,
which spans multiple decades and distinct macroeconomic regimes, the
correlation is essentially zero and the regression slopes are
statistically indistinguishable from zero, even though individual
subperiods---such as the high-inflation 1970s or the low-inflation,
low-unemployment years before the Great Recession---appear quite
different in the plots. Taken together, the analysis suggests that any
Phillips-curve-type trade-off in the U.S. data is period-specific rather
than permanent, and that the strength and sign of the
inflation--unemployment relationship depend critically on the time
window and the measure of inflation under consideration.

\end{document}
